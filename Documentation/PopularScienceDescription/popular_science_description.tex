\documentclass{article}
\usepackage[utf8]{inputenc}
\usepackage{amsmath}
\usepackage{amsthm}
\usepackage{amsfonts}
\usepackage{parskip}
\setlength{\parindent}{0em}
\setlength{\parskip}{1em}
\usepackage{algorithm}
\usepackage{algpseudocode}
\usepackage{graphicx}
\usepackage{systeme}
\usepackage{tikz}
\usepackage[
backend=biber,
style=ieee,
sorting=ynt
]{biblatex}
\usepackage{pdfpages}

\title{ Improving brain simulations by using heuristics \\ Popular science description - Degree Project in Computer Science, DD150X}
\author{Daniel Benedí \\ Supervisor: Alexander Kozlov}
\date{May 2022}


\begin{document}
\maketitle

The human brain is a very complex organ. Unfortunately, it can develop disorders
such as Alzheimer's disease and dementia. The better neuroscientists understand
how the brain and its diseases work, the better they will be able to develop 
treatments and preventions. To do so, scientists make use of simulations of the 
brain. However, the brain is very complex and it is impossible to simulate it
utterly with today's technology. Therefore, the bigger the region scientists can
reproduce, the greater understanding of the brain they will obtain and the better 
treatments they will produce.

When a neuroscientist wants to research a region of the brain, they will obtain 
an accurate representation of the types of neurons. Then they will recreate the 
neurone network defining the connections between neurons. Finally, they will 
simulate the signals exchanges between neurons. The task of establishing the 
connections is really time-consuming and computationally hard. To spped up this
task, we tried to bring some heuristics used in computer graphics for ray tracing,
but they had no effect on the execution time. If there is more research in this 
field, it will promote better and more efficient healthcare for everyone.

\end{document}

