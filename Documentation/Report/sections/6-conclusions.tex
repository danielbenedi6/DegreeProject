\chapter{Conclusion}
\label{chapter:conclusion}

\section{Future Work}
The next step in this research is to optimise the query algorithm to discard some branches of the $k$-d tree if the length in that dimension is too far from the query point. Also, future research must experiment with the number of $k$-d trees, because it will vary in performance. Finally, there must be more research on heuristics since it has helped in other fields and usually improves the computational cost. In other fields, researchers are introducing distributed computing due to its scalability compared to the price. So a distributed $k$-d tree must be taken into account for bigger and more realistic cases.

\section{Conclusion}
This study has investigated the performance of four heuristics for building a $k$-d tree, surface area heuristic, curve complexity heuristic, median of the hyperplane with maximum variance and minimum variance union. The results show that non of the heuristics increase the performance for the touchpoint task for one $k$-d tree per neuron. Thus, more tests are needed on different values of neurons per $k$-d tree, larger populations of neurons and mixed types of neurons, in order to find the most suitable one for whole-brain simulations.